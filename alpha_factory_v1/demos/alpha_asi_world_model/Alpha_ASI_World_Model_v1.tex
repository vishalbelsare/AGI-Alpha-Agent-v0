\documentclass{article}
\usepackage[margin=1in]{geometry}
\usepackage{hyperref}
\usepackage{enumitem}
\usepackage{xcolor}

\begin{document}

\title{\textbf{Alpha‑Factory v1: Multi‑Agent AGENTIC $\alpha$‑AGI World Model Demo}}
\date{}
\maketitle

\section*{Introduction and Objectives}

The \textbf{Alpha‑Factory v1} demo showcases a \textbf{large--scale foundation world model} driven by a \textbf{constellation of autonomous agents}. 
The goal is to generate \textbf{diverse synthetic environments} and train \textbf{general, robust agents} on an open--ended curriculum, inching toward $\alpha$‑ASI (artificial super‑intelligence). 
This project builds on Montreal.AI's existing \textit{AGI‑Alpha‑Agent‑v0} code‑base and incorporates cutting‑edge ideas in AI research. 
We aim for a \textbf{production‑ready, flawless implementation} that can be deployed by non‑technical users, demonstrating emergent general intelligence through multi‑agent collaboration.

\subsection*{Key objectives}

\begin{itemize}[leftmargin=*]
  \item \textbf{Multi‑Agent Orchestration:} Leverage at least five integrated agents from the Alpha‑Factory suite working in concert (planner, learner, evaluator, environment‑generator, etc.).
  \item \textbf{Open‑Ended World Generation:} Autonomously create and evolve diverse training environments (virtual worlds, tasks, simulations) to continually challenge and improve agents.
  \item \textbf{Advanced Training Loops:} Implement both \textit{MuZero}--style model‑based learning and \textit{POET}--style co‑evolution of environments and agents for robust skill acquisition.
  \item \textbf{Integration of AI Protocols:} Use OpenAI's Agents SDK, Google's ADK, Agent2Agent (A2A) protocol, and Anthropic's Model Context Protocol (MCP) to enhance interoperability, security, and learning capabilities.
  \item \textbf{User‑Friendly Deployment:} Provide a simple UI/REST API for non‑experts, a CLI for developers, and containerized deployment (Docker/K8s) for easy scaling.
  \item \textbf{Antifragility \& Robustness:} Design the system to become \textit{more resilient under stress} (antifragile) and \textit{secure by default}, with broad applicability across industries --- \textit{Outlearn, Outthink, Outdesign, Outstrategize, Outexecute} in any domain.
\end{itemize}

\section*{Architecture Overview}

\textbf{Alpha‑Factory v1} is an \textbf{antifragile multi‑agent architecture}. It consists of an orchestrator and a network of specialized agents, all built on the existing code‑base's patterns and extended for this demo. Each agent has a distinct role, and together they form an \textbf{agentic $\alpha$‑AGI system} where the whole is greater than the sum of parts:

\begin{itemize}[leftmargin=*]
  \item \textbf{Orchestrator (Macro‑Sentinel):} The central brain coordinating all agents. It spawns agents, assigns tasks, and manages the iterative training cycles. The orchestrator uses \textit{A2A protocol} for agent communication, enabling independent modules to share goals and state regardless of framework or language. 
  \item \textbf{Environment Generator Agent:} Creates synthetic world models and tasks. Drawing from \textit{POET} principles, it generates a diverse and ever‑expanding curriculum of environments, using quality‑diversity (QD) algorithms to introduce novelty and complexity.
  \item \textbf{Learning/Planning Agent:} A reinforcement‑learning agent that interacts with environments using a MuZero‑style approach (learned model + MCTS planning), building an internal world model to support look‑ahead planning and generalisation.
  \item \textbf{Curriculum \& Evaluation Agent:} Monitors performance and adapts the curriculum, inspired by POET co‑evolution and the ``Era of Experience'' paradigm. It ensures the training data evolves as the agent becomes stronger.
  \item \textbf{Knowledge/Memory Agent:} Utilises MCP to provide memory and external context to other agents, storing learned skills and summarising past experience via optional LLM tools.
  \item \textbf{Control \& Safety Agent:} Enforces constraints, intervenes on unsafe behaviour, and implements best‑practice safeguards from OpenAI's \textit{Practical Guide to Building Agents}.
  \item \textbf{Interface Agents:} Expose REST/WS APIs and dashboards, and translate user intents to orchestrator actions.
\end{itemize}

All agents leverage the Alpha‑Factory backend, with modular design and standardised communication (A2A, MCP).

\section*{Open‑Ended Environment Generation}

A core innovation in this demo is its \textbf{world generator} that produces an endless stream of varied environments, addressing Clune's \textit{third pillar} of AI‑GAs --- automatically generating effective learning environments.

\subsection*{Diverse Synthetic Worlds}

\begin{enumerate}[leftmargin=*]
  \item Start with simple base environments and an untrained agent.
  \item Mutate or perturb environments periodically, producing new challenges.
  \item Evaluate via minimal‑criterion checks; shelve impossible tasks, advance solvable ones.
  \item Maintain novelty by diversity metrics; prevent repetition and encourage generalisation.
\end{enumerate}

\subsection*{Curriculum and Experience Engine}

The curriculum agent automatically adjusts difficulty, provides shaping rewards or stepping‑stone tasks, revisits older tasks to prevent catastrophic forgetting, and thus acts as an \textit{algorithmic mentor} that ``learns how to teach''.

\section*{MuZero‑Style Learning and Planning}

Within each environment, agents train via a MuZero loop:

\begin{itemize}[leftmargin=*]
  \item \textbf{Learned World Model:} Predicts value, reward, and policy; encodes latent state.
  \item \textbf{MCTS:} Performs look‑ahead search in the learned model.
  \item \textbf{No Hard‑Coded Rules:} Dynamics inferred from experience.
  \item \textbf{Training Loop:} Self‑play / experience replay updates representation, dynamics, and prediction networks continually.
\end{itemize}

\section*{POET‑Style Co‑Evolution Loop}

\begin{enumerate}[leftmargin=*]
  \item \textbf{Environment Proposal} with minimal‑criterion filtering.
  \item \textbf{Agent Adaptation} via targeted training or fine‑tuning.
  \item \textbf{Incorporation \& Transfer} of new skills/environment into the curriculum.
  \item Repeat indefinitely, yielding open‑ended improvement and emergent capabilities.
\end{enumerate}

\section*{Integration of Advanced AI Frameworks}

We integrate:

\begin{itemize}[leftmargin=*]
  \item \textbf{OpenAI Agents SDK} (functions/tools, ReAct planning).
  \item \textbf{Google ADK} (standardised micro‑service lifecycle).
  \item \textbf{Agent2Agent (A2A)} for inter‑agent messaging.
  \item \textbf{Model Context Protocol (MCP)} for safe context injection.
\end{itemize}

All cloud dependencies are optional; the demo runs fully offline by default.

\section*{User Interaction and Deployment}

\subsection*{Interfaces}

\begin{itemize}[leftmargin=*]
  \item Web UI Dashboard (Streamlit/React).
  \item REST API (FastAPI).
  \item CLI for power users.
  \item Configurable via YAML/dataclass.
\end{itemize}

\subsection*{Deployment Options}

\begin{itemize}[leftmargin=*]
  \item Local Python execution.
  \item Docker container.
  \item Kubernetes Helm chart for cloud scale.
  \item Security best‑practices: sandboxing, resource limits, token auth.
\end{itemize}

\section*{Ensuring Antifragility, Security, and Robust Intelligence}

\begin{itemize}[leftmargin=*]
  \item \textbf{Antifragility:} Failures trigger self‑analysis and curriculum adjustment.
  \item \textbf{Robustness:} Automated tests, perturbation stress‑tests, redundancy across agents.
  \item \textbf{Security:} Role scoping, API sanitisation, encrypted data at rest.
  \item \textbf{Emergent Behaviour:} Meta‑learning and cross‑domain transfer evaluated continually.
\end{itemize}

\section*{Conclusion and Deliverables}

\begin{itemize}[leftmargin=*]
  \item Complete, documented code‑base.
  \item Detailed user guide and architecture diagrams.
  \item Test‑suite ensuring flawless functionality.
  \item Preset demo scenarios and an emergent behaviour showcase log.
\end{itemize}

The Alpha‑Factory v1 demo forms a \textbf{comprehensive foundation for $\alpha$‑AGI research}, embodying the principles of experience‑driven learning and open‑ended self‑improvement.

\end{document}
